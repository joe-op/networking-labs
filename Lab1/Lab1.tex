%% LyX 2.1.4 created this file.  For more info, see http://www.lyx.org/.
%% Do not edit unless you really know what you are doing.
\documentclass[english]{article}
\usepackage[T1]{fontenc}
\usepackage[latin9]{inputenc}

\makeatletter

%%%%%%%%%%%%%%%%%%%%%%%%%%%%%% LyX specific LaTeX commands.
%% Because html converters don't know tabularnewline
\providecommand{\tabularnewline}{\\}

%%%%%%%%%%%%%%%%%%%%%%%%%%%%%% User specified LaTeX commands.
\date{}

\makeatother

\usepackage{babel}
\begin{document}

\title{Lab 1}


\author{Joe Opseth \& Becca Woitas}

\maketitle

\subsection*{FireWire vs USB}

FireWire and USB are two standards for connecting devices. Firewire
is less common and does not require a PC to act as the host.

\begin{tabular}{|c|c|}
\hline 
FireWire & USB\tabularnewline
\hline 
\hline 
800 Mbps & 480 Mbps (2.0), 5Gbps (3.0)\tabularnewline
\hline 
Peer-to-Peer & Needs PC Host\tabularnewline
\hline 
Up to 33 Volt output & Up to 5 Volt output\tabularnewline
\hline 
Up to 15' Cable & Up to 16' Cable\tabularnewline
\hline 
\end{tabular}


\subsection*{Costs of Memory}

\begin{tabular}{|c|c|c|}
\hline 
Type of Memory & Cost & Cost/GB\tabularnewline
\hline 
\hline 
RAM & 16GB for \$34 & \$2.125\tabularnewline
\hline 
SSD & 512GB for \$120 & \$0.23\tabularnewline
\hline 
SATA & 500GB for \$20 & \$0.04\tabularnewline
\hline 
\end{tabular}

Source: newegg.com


\subsection*{CSEL Configuration of an IDE Bus}

CSEL or cable select configuration is where the 'master' hard drive
is the one plugged into the master connector on the motherboard, without
needing to set jumpers.


\subsection*{Binary MAC Address to Hex}

0000 0000 0000 0100 0111 0110 1011 1010 1100 0010 0011 0001 => 000476BAC231


\subsection*{IP Address vs MAC Address}

IP is user configured, whereas the MAC address is set on the Network
Interface Card. MAC address is used on LANs and IP is used on the
Internet.


\subsection*{TCP Sequence Number and Acknowledge Numbers}

The sequence number indicates where the current data belongs in the
entire data stream. The acknowledgement number is the next packet
that the receiver is expecting, which indicates that previous packets
have been received.


\subsection*{ARP}

ARP (Address Request Protocol) is a standard that computers on the
same LAN use to discover each other's MAC addresses. It involves broadcasting
a signal with destination MAC address all 1's, requesting the computer
with a certain IP address; every computer will get this message and
the computer with matching IP address will respond.
\end{document}
